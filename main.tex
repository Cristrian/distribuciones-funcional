\documentclass[letter,12pt]{article}
\usepackage{presets}
%opening
\title{Teoría de Distribución}
\author{Cristian Camilo Triana García}

\begin{document}

\maketitle

\begin{abstract}\normalsize
Estudiar teoría de distribución
\end{abstract}




\section{Introducción}
La teoría de distribuciones se puede ver como una extensión de la diferenciación,
de la misma manera que podemos ver la teoría de la medida como una extensión de la
integración.

El objetivo de ésta teoría es extender la clase de funciones derivables, extender
nuestra definición de diferenciación. Para conseguir esto, necesitamos construir
una clase de funciones que cumplan los siguientes requisitos:

\end{document}
