\documentclass[letter,12pt]{article}
\usepackage{presets}
%opening
\title{Teoría de Distribución}
\author{Cristian Camilo Triana García}

\begin{document}

\maketitle

\begin{abstract}\normalsize
Estudiar teoría de distribución
\end{abstract}




\section{Introducción}
La teoría de distribuciones se puede ver como una extensión de la diferenciación,
de la misma manera que podemos ver la teoría de la medida como una extensión de la
integración.

El objetivo de ésta teoría es extender la clase de funciones derivables, más específicamente
nuestra definición de diferenciación. Para conseguir esto, necesitamos construir
una clase de objetos(que llamaremos \textbf{distribuciones}) que cumplan los siguientes requisitos:

\begin{enumerate}
	\item Toda función continua debe ser una distribución.
	\item La nueva noción de derivada debe coincidir con la anterior.
	\item Las derivadas parciales de toda distribución, deben ser también distribuciones.
	\item Las propiedades usuales del cálculo deben conservarse.
	\item Es necesario tener varios teoremas de convergencia para las distribuciones,
		que nos permitan trabajar con límites.
\end{enumerate}

Esta nueva noción de distribuciones es de utilidad para formalizar varias ideas matemáticas que se usaban en el cálculo de manera intuitiva y nada formal, especialmente en la física a la hora de trabajar con ecuaciones diferenciales parciales. 

Consideremos, por ejemplo, la función \textit{escalón de Heaviside} $ H $ definida por:

\begin{equation}
	H(x) = 
	\begin{cases}
		0, \qquad x < 0 \\
		1, \qquad x \ge 0. 
	\end{cases}
\end{equation}

Se dice que la "derivada" de esta función es la función delta de Dirac $ \delta(x) $, la cual se anula en todo su dominio salvo en el origen, en donde su valor es tan grande que tenemos que:
\begin{equation}
	\int_{-\infty}^{+\infty}\delta(x) dx = 1.
\end{equation} 
Esta "función" y sus "derivadas" es usada con bastante frecuencia en la matemática.
Para definir la función delta de Dirac, es necesario utilizar la noción de medida.

\end{document}
