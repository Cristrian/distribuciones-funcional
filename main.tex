\documentclass[letter,12pt]{article}
\usepackage{graphicx}
\usepackage[spanish]{babel}
\usepackage{amsthm}
\usepackage{tikz}
\usepackage{bm}
\usepackage{amsmath}
\usepackage{amsfonts}
\usepackage{amssymb}
\usepackage{mathrsfs}
\usepackage[scr=rsfso,cal=zapfc,frak=euler,bb=ams]{mathalfa}
\usepackage{biblatex}

%Commands
\newcommand\norm[1]{\lVert#1\rVert}
%---Definitions
\newtheorem{lemma}{Lema}
\newtheorem{remark}{Nota}
\newtheorem{definition}{Definición}
\newtheorem{theorem}{Teorema}
\newtheorem{example}[theorem]{Ejemplo}
%bib import
\addbibresource{bibliografia.bib}


%opening
\title{Teoría de Distribución}
\author{Cristian Camilo Triana García}
\begin{document}

\maketitle

\begin{abstract}\normalsize
Estudiar teoría de distribución
\end{abstract}




\section{Introducción}
La teoría de distribuciones se puede ver como una extensión de la diferenciación,
de la misma manera que podemos ver la teoría de la medida como una extensión de la
integración.

El objetivo de ésta teoría es extender la clase de funciones derivables, más específicamente
nuestra definición de diferenciación. Para conseguir esto, necesitamos construir
una clase de objetos(que llamaremos \textbf{distribuciones}) que cumplan los siguientes requisitos:

\begin{enumerate}
	\item Toda función continua debe ser una distribución.
	\item La nueva noción de derivada debe coincidir con la anterior.
	\item Las derivadas parciales de toda distribución, deben ser también distribuciones.
	\item Las propiedades usuales del cálculo deben conservarse.
	\item Es necesario tener varios teoremas de convergencia para las distribuciones,
		que nos permitan trabajar con límites.
\end{enumerate}

Esta nueva noción de distribuciones es de utilidad para formalizar varias ideas matemáticas que se usaban en el cálculo de manera intuitiva y nada formal, especialmente en la física a la hora de trabajar con ecuaciones diferenciales parciales. 

Consideremos, por ejemplo, la función \textit{escalón de Heaviside} $ H $ definida por:

\begin{equation}
	H(x) = 
	\begin{cases}
		0, \qquad x < 0 \\
		1, \qquad x \ge 0. 
	\end{cases}
\end{equation}

Se dice que la "derivada" de esta función es la función delta de Dirac $ \delta(x) $, la cual se anula en todo su dominio salvo en el origen, en donde su valor es tan grande que tenemos que:
\begin{equation}
	\int_{-\infty}^{+\infty}\delta(x) dx = 1.
\end{equation} 
Esta ``función'' y sus ``derivadas'' es usada con bastante frecuencia en la matemática.
Para definir la función delta de Dirac formalmente, es necesario utilizar la noción de medida.

Las distribuciones pueden ser definidas en $ \mathbb{R}^n $, sin embargo, antes de proceder con las definiciones, vamos a ver el caso de $ n =1 $ para entender un poco la lógica detrás.

Cuando tenemos una función real $ f $, la relación que nos interesa es la que se da
punto a punto, es decir, $ f(x) $ para $ x \in \mathbb{R} $, 
en distintas aplicaciones lo que interesa no es la el valor puntual de la función 
$ f(x) $, sino que nos interesa es su comportamiento al ser integrada, ya sea sola 
o en producto con otra función. Esto es algo muy común en la física, cuando 
por ejemplo un fenómeno es modelado por dos funciones cuyo producto nos
proporciona información valiosa, o cuando se desean obtener promedios. 
Así, obtenemos integrales del siguiente estilo:

\begin{equation}
	\int{f(x)\phi(x)}dx.
\end{equation}

Teniendo en cuenta esto, podemos pensar en asignar a $ f $, el valor numérico obtenido 
al resolver la integral, en lugar de $ f(x) $.

La función $ \phi $ es llamada función \textit{test}, más adelante veremos cómo se hace 
su elección.

Sea $ \mathscr{D}(\mathbb{R}) $ el espacio vectorial de todas las funciones 
$ \phi \in \mathbf{C}^\infty (\mathbb{R}) $ (infinitamente diferenciables), cuyo soporte
$ \left\{ x\in \mathbb{R}| \phi(x) \neq 0 \right\} $ es compacto.

Ahora, si además $ f $ es diferenciable, tenemos que
\begin{equation}\label{int:first}
	\int{f'\phi} = - \int{f\phi '}.
\end{equation}
Y si $ f $ es infinitamente diferenciable, entonces
\begin{equation}\label{int:second}
	\int{f^{(k)}\phi} = (-1)^k \int{f\phi^{(k)}}.
\end{equation}

Estos resultados nos muestran el punto clave y la motivación para nuestras definiciones,
si nos fijamos, en el lado derecho de las ecuaciones \ref{int:first} y
\ref{int:second}, podemos ver que la derivada de $ f $ no es tenida en cuenta.
Así, lo que queremos es construir un funcional $ F: \mathscr{D} \in \mathbb{R} $ ´tal 
que
\begin{equation}
	F'(\phi) = -F(\phi').
\end{equation}

\section{Distribuciones y sus propiedades básicas.}
Cómo lo que queremos es definir un funcional continuo, primero necesitamos construir 
una topología, en la cual se validará la continuidad.

El trabajo que vamos a desarrollar ahora es para $ \mathbb{R}^n $. En las 
definiciones siguientes $ \Omega $ es un subconjunto abierto de 
$ \mathbb{R}^n $. Por $ \partial_{k} $ vamos a denotar la derivada parcial
$ \partial/\partial{x_k} $. El operador de orden mayor 
$ \partial^{p_1}_{1} \partial^{p_2}_{2} \cdots \partial^{p_n}_{n} $ lo denotamos
por $ \partial^p $ y $ |p| = p_1 + \cdots + p_n $ es el orden de $ \partial^p $

Denotamos por $ \mathscr{D}(\Omega) $ el conjunto formado por las funciones 
$ \phi: \mathbb{R}^n \to \mathbb{R} $ cuyo soporte es compacto, y que todas sus
derivadas parciales existen. Es decir, $ \phi \in \mathscr{D}(\Omega) $ si, y sólo si
$ \phi \in \mathbf{C}^{\infty}(\Omega) $ y su soporte 
$ \left\{ x| \phi(x) \neq 0 \right\} $ es compacto en $ \Omega $.
Dado un subconjunto $ S $ de $ \Omega $, $ \mathscr{D}(\Omega, S) $ denota el 
subespacio de $ \mathscr{D}(\Omega) $ compuesto por las funciones 
$ \phi\in\mathscr{D}(\Omega) $ cuyo soporte está contenido en $ S $.

Ahora, vamos a definir las siguientes normas
\begin{equation}
 	\norm{\phi}^{S}_{m} := 
 	\sup{ \left\{ |\partial^p{\phi(x)}|: x\in S, \quad |p| < m  \right\} }
\end{equation}

Las normas que acabamos de ver nos sirven para definir una topología completa, 
metrizable y localmente convexa en $ \mathscr{D}(\Omega, S) $, lo que hace 
$ \mathscr{D}(\Omega, S) $ un espacio de Frechet.


\printbibliography
\end{document}
